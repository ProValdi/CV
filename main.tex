% -- Encoding UTF-8 without BOM
% -- XeLaTeX => PDF (BIBER)


\documentclass[]{cv-style}          % Add 'print' as an option into the square bracket to remove colours from this template for printing. 
                                    % Add 'espanol' as an option into the square bracket to change the date format of the Last Updated Text

\sethyphenation[variant=british]{english}{} % Add words between the {} to avoid them to be cut 

\usepackage{tikz}

\newcommand{\roundpic}[4][]{
  \tikz\node [circle, minimum width = #2,
    path picture = {
      \node [#1] at (path picture bounding box.center) {
        \includegraphics[width=#3]{#4}};
    }] {};}


\begin{document}

\header{Vladimir}{ Prokhorov}       % Your name
\lastupdated

%----------------------------------------------------------------------------------------
%	SIDEBAR SECTION  -- In the aside, each new line forces a line break
%----------------------------------------------------------------------------------------

\begin{aside}

\vspace{30pt}
\begin{minipage}[H]{\linewidth}
\roundpic[xshift=0.12cm,yshift=-0.6cm]{3.5cm}{6.3cm}{main.jpg}
\end{minipage}

%
\cvsection{Contact}
~
\icontext{EnvelopeSquare}{14}{{\mbox{prokhorov.va}
@phystech.edu}}{black}\\[6pt]
~
\icontext{Github}{14}{{\mbox{github.com/ProValdi}}}{black}\\[6pt]
~
\icontext{Linkedin}{14}{{linkedin.com/in/provaldi}}{black}\\[6pt]
~
\icontext{Phone}{14}{{+7(915)0249633}}{black}\\[6pt]
~
%
\cvsection{Languages}
~
English Intermediate
%
\cvsection{Skills}
~
\cvskill{Java {\color{red} $\varheartsuit$}} {1} \\[-2pt]
~
\cvskill{SQL} {0.6} \\[-2pt]
~
\cvskill{Git} {0.9} \\[-2pt]
~
\cvskill{Docker} {0.6} \\[-2pt]
~
\cvskill{Spring} {0.7} \\[-2pt]
~
\cvskill{Linux} {0.6} \\[-2pt]
~
\cvskill{Quarkus} {0.7} \\[-2pt]
~


%
\end{aside}

%----------------------------------------------------------------------------------------
%	SKILLS SECTION
%----------------------------------------------------------------------------------------


%----------------------------------------------------------------------------------------
%	EDUCATION SECTION
%----------------------------------------------------------------------------------------

\section{Education}

\begin{entrylist}
%------------------------------------------------
\entry
{2018 -- 2024}
{{\normalfont Department of Radio Engineering and Cybernetics,\newline Bachelor's + Master's degree in Applied Mathematics and Physics}}
{\\ Moscow Institute of Physics and Technology (MIPT)  |  Dolgoprudny, Russia \\Department of Information Systems and Networks | Netcracker}
{\vspace{-0.3cm}}
%------------------------------------------------
\end{entrylist}


%----------------------------------------------------------------------------------------
%	WORK EXPERIENCE SECTION
%----------------------------------------------------------------------------------------

\section{Experience}

\begin{entrylist}
%------------------------------------------------

\entry
    {Jul 2021 -- Now}
    {Netcracker}
    {}
    {\jobtitle{Software Developer}
    \\- Developed from scratch scalable deployable Quarkus-based containerized custom Java microservice featuring job scheduling (using Quartz) with connection to existing kafka topic (maas) for business data patching (using REST clients of other microservices). 
    \\- Developed Spring-based Java microservice for assembling and converting business data via Jooq providing REST API for other dependent microservices of our project ecosystem.
    \\ - Stabilized CI environment (real-time bug-fixing)
    \\ - Developed new project's GraphQL API.
    \\ - Developed unit tests (Junit5) for lots of microservices inside our project ecosystem.
    \\ - Created safe business datafixes using SQL language (direct production data impact).   
    
}

\end{entrylist}



My working expertise consists of many other technologies and frameworks that I have used through the whole 3-year coding experience. I've touched Redis and Kassandra for caching purposes, created a bunch of Docker images and able to handle them inside K8S cluster, developed a simple Netty-based client-server chat and I also have a good experience with Git. I know Confluence and Jira (used both as a part of my daily work) for referencing documentation and task planning purposes. I worked with ElasticSearch. I understand what is Jenkins and how CI/CD works. I worked with Angular and TypeScript (and all front-end stuff). I omit some other technologies and in order to keep the long story short: I feel free to learn and use new technologies that I’ve never seen before.  



\section{Volunteering}

\begin{entrylist}

\entrySkoltech
    {2018--2021}
    {Experience As a Teacher}
    {\jobtitle{}\\
Taught general physics for pupils at MIPT preparation courses.}

\end{entrylist}


%----------------------------------------------------------------------------------------
%	SOFT SKILLS SECTION
%----------------------------------------------------------------------------------------

%\section{Soft Skills}
%{\vspace{0.1cm}}
%I actually love working in team and interacting with other people. I'm not afraid to be creative and sometimes it leads to reinventing the wheel. Responsible and hardworking. I try to find the right approach to different people while working on problem.

%----------------------------------------------------------------------------------------
%		INTERESTS SECTION
%----------------------------------------------------------------------------------------

% \section{Electrical Engineering Experience}
% {\vspace{0.005cm}}
% During hole my conscious life I tend to create different electrical devices and for the last 8 years I have been improving my skills in creating those. Starting with the very

\section{Interests}
%------------------------------------------------
%\entry
%{Проф-ные}
%{- {\normalfont Continue learning, work and develop a career in what I'm passionate about, which are %all the information technologies. }}
%{}
%{\vspace{-0.3cm}}
%------------------------------------------------
{\vspace{0.05cm}}
Usually this section is much more about me, because I am not only a Java programmer, although my total experience in it is about 10 years (3 of which are in enterprise).
I currently hold a part-time position as a junior researcher at the Kotelnikov's Research Institute as a part of my diploma work. Participated in the En&T Conference at Nov 2023 and later in 66th All-Russian Scientific Conference MIPT at Mar 2024. I develop PCB which can measure distance using UWB rays. During hole my conscious life I tend to create different electrical devices and for the last 8 years I have been improving my skills in creating those. Starting with the very basics and Arduino, ending with STM32 and digital processing using FPGAs.
I also can share this experience:

\begin{entrylist}

\entrySkoltech
{2023, Nov\\2024, Mar}
{\text{En\&T} Conference}
{
\\
A report was presented on the topic "Positioning using UWB signals" regarding scientific research financed by RSF (Russian Science Foundation) \text{№23-29-00883}
}


\entrySkoltech
{2021, Aug}
{Skoltech Summer Internship}
{
\\
Developed independent load balancer for iPerf-based 5G speedtest service by working with my colleague using Python Flask server in combination with Docker containerization and Swagger API.
}



\entry
    {2020, Apr}
    {NTI Hackathon}
    {}
    {\jobtitle{Student stream, Wireless Technologies Profile}\\
Won first place out of 8 teams in hackathon at NTI by working with five colleagues to develop a noise-resistant algotirthm for optimal data transmission over a noisy channel.
    }
    
    

\end{entrylist}


%------------------------------------------------

\end{document}