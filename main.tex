% -- Encoding UTF-8 without BOM
% -- XeLaTeX => PDF (BIBER)


\documentclass[]{cv-style}          % Add 'print' as an option into the square bracket to remove colours from this template for printing. 
                                    % Add 'espanol' as an option into the square bracket to change the date format of the Last Updated Text

\sethyphenation[variant=british]{english}{} % Add words between the {} to avoid them to be cut 

\usepackage{tikz}

\newcommand{\roundpic}[4][]{
  \tikz\node [circle, minimum width = #2,
    path picture = {
      \node [#1] at (path picture bounding box.center) {
        \includegraphics[width=#3]{#4}};
    }] {};}


\begin{document}

\header{Vladimir}{ Prokhorov}       % Your name
\lastupdated

%----------------------------------------------------------------------------------------
%	SIDEBAR SECTION  -- In the aside, each new line forces a line break
%----------------------------------------------------------------------------------------

\begin{aside}

\vspace{30pt}
\begin{minipage}[H]{\linewidth}
\roundpic[xshift=0.12cm,yshift=-0.6cm]{3.5cm}{6.3cm}{main.jpg}
\end{minipage}

%
\cvsection{Contact}
~
\icontext{EnvelopeSquare}{14}{{prokhorov.va
@phystech.edu}}{black}\\[6pt]
~
\icontext{Github}{14}{{github.com/ProValdi}}{black}\\[6pt]
~
\icontext{Linkedin}{14}{{linkedin.com/in/provaldi}}{black}\\[6pt]
~
\icontext{Phone}{14}{{+7(915)0249633}}{black}\\[6pt]
~
%
\cvsection{Languages}
~
English Intermediate
%
\cvsection{Skills}
~
\cvskill{Java {\color{red} $\varheartsuit$}} {1} \\[-2pt]
~
\cvskill{PostgreSQL} {0.7} \\[-2pt]
~
\cvskill{Git} {0.9} \\[-2pt]
~
\cvskill{Linux} {0.6} \\[-2pt]
~
\cvskill{Spring} {0.7} \\[-2pt]
~


%
\end{aside}

%----------------------------------------------------------------------------------------
%	SKILLS SECTION
%----------------------------------------------------------------------------------------


%----------------------------------------------------------------------------------------
%	EDUCATION SECTION
%----------------------------------------------------------------------------------------

\section{Education}

\begin{entrylist}
%------------------------------------------------
\entry
{2018 -- Now}
{{\normalfont Department of Radio Engineering and Cybernetics,\newline Bachelor's + Master's degree in Applied Mathematics and Physics}}
{\\ Moscow Institute of Physics and Technology (MIPT)  |  Dolgoprudny, Russia \\Department of Information Systems and Networks | Netcracker}
{\vspace{-0.3cm}}
%------------------------------------------------
\end{entrylist}


%----------------------------------------------------------------------------------------
%	WORK EXPERIENCE SECTION
%----------------------------------------------------------------------------------------

\section{Experience}

\begin{entrylist}
%------------------------------------------------

\entry
    {Jul 2021 -- Now}
    {Netcracker}
    {}
    {\jobtitle{Software Developer}\\ 
    - Optimized and structured for better readability the code of an existing microservice written in Java by better organization of object dependencies;
    \\ - Maintained the working dev environments in Kubernetes by deploying vital microservices with properly configured parameters;
    \\ - Configured API for GraphQL queries in terms of new microservice.
    \\ - Developed unit tests for new Java component with high code coverage using JUnit5.\\

- Worked with message brokers: Kafka, RabbitMQ\\
- Developed new microservice using Spring, Maven and Jooq for data fetching from DB. Also work with PostgreSQL\\
- Know Docker and K8S (OpenLens/native dashboard), keycloak\\
- Worked with Swagger, Postman, REST api, GraphQL\\
- I've touched Redis, Kassandra\\
- I know Confluence and Jira, Agile (scrum), Git\\
- Understand what is CI/CD, Jenkins (and how use it), ElasticSearch, Prometheus (both slightly understand)
}

% \entrySkoltech
%     {2021, Aug}
%     {Skoltech Summer Internship}
%     {
%     \\
%     Developed independent load balancer for iPerf-based 5G speedtest service by working with my colleague using Python Flask server in combination with Docker containerization and Swagger API.
%     }

% \entry
%     {2020, Apr}
%     {NTI Hackathon}
%     {}
%     {\jobtitle{Student stream, Wireless Technologies Profile}\\
% Won first place out of 8 teams in hackathon at NTI by working with five colleagues to develop a noise-resistant algotirthm for optimal data transmission over a noisy channel.
%     }

%\entry
%  {2018--Now}
%  {Talento ConCrédito}
%  {Culiacán, Sinaloa}
%  {\jobtitle{Developer}\\
%  Create, Modify, Maintain and deploy API's, in the develop, QA and production environment.
%Our API's help to provide a financial solutions simplify by technology.
%
%Work with the scrum methodology.
%
%%Version control of our projects using git.
%
%Use the Google Data Studio tool to create reports that represent indicators of the systems.
%
%Use Docker to help us through containers to implement tools in a matter of seconds without the need to %install anything. It's multiplatform, which guarantees that the availability of our application works independently of the operating system.}
%------------------------------------------------
%\entry
%  {2018--2018}
%  {ConCrédito }
%  {Cuiacán, Sinaloa}
%  {\jobtitle{Tester}
%  \begin{itemize}
%    \item Analyze systems functional models. 
%    \item Create test cases for the systems.
%    \item Run test cases.
%    \item Create autumatized tests using selenium.
%    \item Black-Box Testing
%    \item Record evidence and raise defects to the developers.
%  \end{itemize}}

\end{entrylist}


\section{Volunteering}

\begin{entrylist}

\entrySkoltech
    {2018--2021}
    {Experience As a Teacher}
    {\jobtitle{}\\
Taught general physics for pupils at MIPT preparation courses.}

\end{entrylist}


%----------------------------------------------------------------------------------------
%	SOFT SKILLS SECTION
%----------------------------------------------------------------------------------------

%\section{Soft Skills}
%{\vspace{0.1cm}}
%I actually love working in team and interacting with other people. I'm not afraid to be creative and sometimes it leads to reinventing the wheel. Responsible and hardworking. I try to find the right approach to different people while working on problem.

%----------------------------------------------------------------------------------------
%		INTERESTS SECTION
%----------------------------------------------------------------------------------------

% \section{Electrical Engineering Experience}
% {\vspace{0.005cm}}
% During hole my conscious life I tend to create different electrical devices and for the last 8 years I have been improving my skills in creating those. Starting with the very

\section{Interests}
%------------------------------------------------
%\entry
%{Проф-ные}
%{- {\normalfont Continue learning, work and develop a career in what I'm passionate about, which are %all the information technologies. }}
%{}
%{\vspace{-0.3cm}}
%------------------------------------------------
{\vspace{0.05cm}}
Usually this section is much more about me, because I am not only a Java-programmer. 
I currently hold a part-time position as a junior researcher at the Kotelnikov's Research Institute as a part of my diploma work. Participated in the En&T Conference at Nov 2023. I develop PCB which can measure distance using UWB rays. During hole my conscious life I tend to create different electrical devices and for the last 8 years I have been improving my skills in creating those. Starting with the very basics and Arduino, ending with STM32 and digital processing using FPGAs.
I also can share this experience:

\begin{entrylist}
\entrySkoltech
{2021, Aug}
{Skoltech Summer Internship}
{
\\
Developed independent load balancer for iPerf-based 5G speedtest service by working with my colleague using Python Flask server in combination with Docker containerization and Swagger API.
}



\entry
    {2020, Apr}
    {NTI Hackathon}
    {}
    {\jobtitle{Student stream, Wireless Technologies Profile}\\
Won first place out of 8 teams in hackathon at NTI by working with five colleagues to develop a noise-resistant algotirthm for optimal data transmission over a noisy channel.
    }

 \end{entrylist}


%------------------------------------------------

\end{document}