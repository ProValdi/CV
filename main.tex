% -- Encoding UTF-8 without BOM
% -- XeLaTeX => PDF (BIBER)

\documentclass[]{cv-style}          % Add 'print' as an option into the square bracket to remove colours from this template for printing. 
                                    % Add 'espanol' as an option into the square bracket to change the date format of the Last Updated Text

\sethyphenation[variant=british]{english}{} % Add words between the {} to avoid them to be cut 

\begin{document}

\header{Прохоров}{ Владимир}           % Your name
\lastupdated

%----------------------------------------------------------------------------------------
%	SIDEBAR SECTION  -- In the aside, each new line forces a line break
%----------------------------------------------------------------------------------------

\begin{aside}
%
\section{contact}
Москва
~
+7(915) 024 9633
~
github.com/ProValdi
~
prokhorov.va
@phystech.edu
%
\section{languages}
Русский
English Intermediate
Deutsch Beginner
%
\section{programming}
{\color{red} $\varheartsuit$} Java,
C
%
\end{aside}

%----------------------------------------------------------------------------------------
%	SKILLS SECTION
%----------------------------------------------------------------------------------------

\section{Skills}
  \vspace{-0.4cm}
Java, IntelliJ IDEA, Git, Maven, Angular, Spring Boot, Hibernate, Docker, C, Fusion 360, \LaTeX{}, PCB design, Arduino, STM32, Ubuntu, Windows.

%----------------------------------------------------------------------------------------
%	WORK EXPERIENCE SECTION
%----------------------------------------------------------------------------------------

\section{Experience}

\begin{entrylist}
%------------------------------------------------

\entry
    {2020--2021}
    {Учебный Центр Netcracker}
    {}
    {\jobtitle{}\\ 
Учёба в УЦ летом 2020: Java SE и XML, Oracle SQL. Реализация проекта в команде аналитиков и разработчиков: Java back-end с использованием системы сборки Maven, БД PostgreSQL, Hibernate и Spring Boot. Front-end с использованием фреймворка Angular. Confluence + система управления проектами Jira. Система контроля версий Git, хранилище GitHub. Получил опыт создания Docker-контейнеров с последующим развёртыванием на AWS.}

\entry
    {2019--2020}
    {Олимпиада КД НТИ 2019/2020}
    {}
    {\jobtitle{}\\
Студенческий трек, Технологии Беспроводной связи. Финалист, 2-е место.

Кодирование и передача информации по зашумлённому каналу, исследование системы.}


\entry
    {2018--2021}
    {Опыт преподавания}
    {}
    {\jobtitle{}\\
Преподавал физику на выездной Летней Экологической Школе, а также на очном отделении ЗФТШ.}

%\entry
%  {2018--Now}
%  {Talento ConCrédito}
%  {Culiacán, Sinaloa}
%  {\jobtitle{Developer}\\
%  Create, Modify, Maintain and deploy API's, in the develop, QA and production environment.
%Our API's help to provide a financial solutions simplify by technology.
%
%Work with the scrum methodology.
%
%%Version control of our projects using git.
%
%Use the Google Data Studio tool to create reports that represent indicators of the systems.
%
%Use Docker to help us through containers to implement tools in a matter of seconds without the need to %install anything. It's multiplatform, which guarantees that the availability of our application works independently of the operating system.}
%------------------------------------------------
%\entry
%  {2018--2018}
%  {ConCrédito }
%  {Cuiacán, Sinaloa}
%  {\jobtitle{Tester}
%  \begin{itemize}
%    \item Analyze systems functional models. 
%    \item Create test cases for the systems.
%    \item Run test cases.
%    \item Create autumatized tests using selenium.
%    \item Black-Box Testing
%    \item Record evidence and raise defects to the developers.
%  \end{itemize}}

\end{entrylist}

%----------------------------------------------------------------------------------------
%	EDUCATION SECTION
%----------------------------------------------------------------------------------------

\section{Education}

\begin{entrylist}
%------------------------------------------------
\entry
{2018--Now}
{{\normalfont ФРТК, Прикладная математика и физика [3 курс]}}
{\\Московский Физико-Технический Институт  |  Долгопрудный, Россия \\Кафедра Информационных Систем и Сетей | Netcracker}
{\vspace{-0.3cm}}
%------------------------------------------------
\end{entrylist}

%----------------------------------------------------------------------------------------
%	SOFT SKILLS SECTION
%----------------------------------------------------------------------------------------

\section{Soft Skills}
  \vspace{-0.4cm}
Хорошо слушаю и слышу, обладаю эмпатией, люблю учиться и узнавать новое, дружу с креативностью. Умею в логическое мышление, стремлюсь к созданию, легко иду на контакт. Способен много работать, а также изобретать велосипеды в процессе обучения. Не обладаю умением мгновенно придумывать решение задачи - люблю подумать.

%----------------------------------------------------------------------------------------
%		INTERESTS SECTION
%----------------------------------------------------------------------------------------

\section{Interests}

\begin{entrylist}
%------------------------------------------------
%\entry
%{Проф-ные}
%{- {\normalfont Continue learning, work and develop a career in what I'm passionate about, which are %all the information technologies. }}
%{}
%{\vspace{-0.3cm}}
%------------------------------------------------
\entry
{Професс.}
{- {\normalfont
Язык программирования Java, фреймворк Spring Boot, Angular, работа в команде, Git. }}
{}
{}
\entry
{Личные}
{- {\normalfont 
Заинтересован в создании электронных устройств. Реализовал такие проекты, как часы с механической развёрткой, катушка Теслы, датчик наклона со световой индикацией и др. В проектах использую микроконтроллеры STM32. Также писал графические приложения на языке Java, чаще игры, в последнее время с использованием библиотеки JavaFX. Писал 3D рендер на Java. Знаком с разработкой приложений под Android. Заинтересован в интернете вещей. Увлечён художественной литературой и спортом: любительский футбол и лёгкая атлетика (из-за коронавируса не очень увлечён). Недавно начал изучение Немецкого языка.}}
{}
{}
%------------------------------------------------
\end{entrylist}



\end{document}