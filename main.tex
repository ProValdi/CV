% -- Encoding UTF-8 without BOM
% -- XeLaTeX => PDF (BIBER)


\documentclass[]{cv-style}          % Add 'print' as an option into the square bracket to remove colours from this template for printing. 
                                    % Add 'espanol' as an option into the square bracket to change the date format of the Last Updated Text

\sethyphenation[variant=british]{english}{} % Add words between the {} to avoid them to be cut 

\usepackage{tikz}

\newcommand{\roundpic}[4][]{
  \tikz\node [circle, minimum width = #2,
    path picture = {
      \node [#1] at (path picture bounding box.center) {
        \includegraphics[width=#3]{#4}};
    }] {};}


\begin{document}

\header{Vladimir}{ Prokhorov}       % Your name
\lastupdated

%----------------------------------------------------------------------------------------
%   SIDEBAR SECTION  -- In the aside, each new line forces a line break
%----------------------------------------------------------------------------------------

\begin{aside}

\vspace{30pt}
\begin{minipage}[H]{\linewidth}
\roundpic[xshift=0.12cm,yshift=-0.6cm]{3.5cm}{6.3cm}{main.jpg}
\end{minipage}

%
\cvsection{Contact}
~
\icontext{EnvelopeSquare}{14}{{prokhorov.va
@phystech.edu}}{black}\\[6pt]
~
\icontext{Github}{14}{{github.com/ProValdi}}{black}\\[6pt]
~
\icontext{Linkedin}{14}{{linkedin.com/in/provaldi}}{black}\\[6pt]
~
\icontext{Phone}{14}{{+7(915)0249633}}{black}\\[6pt]
~
%
\cvsection{Languages}
~
English Intermediate
~
Deutsch Beginner
%
\cvsection{Skills}
~
~
\cvskill{C} {0.7} \\[-2pt]
~
\cvskill{C++} {0.3} \\[-2pt]
~
\cvskill{Linux} {0.6} \\[-2pt]
~
\cvskill{STM32} {0.9} \\[-2pt]
~
\cvskill{Java} {1} \\[-2pt]
~
\cvskill{Python} {0.45} \\[-2pt]
~
\cvskill{Git} {0.7} \\[-2pt]
~


%
\end{aside}

%----------------------------------------------------------------------------------------
%   SKILLS SECTION
%----------------------------------------------------------------------------------------


%----------------------------------------------------------------------------------------
%   EDUCATION SECTION
%----------------------------------------------------------------------------------------

\section{Education}

\begin{entrylist}
%------------------------------------------------
\entry
{2018 -- 2022 Jul}
{{\normalfont Department of Radio Engineering and Cybernetics,\newline Bachelor's degree in Applied Mathematics and Physics}}
{\\ Moscow Institute of Physics and Technology (MIPT)  |  Dolgoprudny, Russia \\Department of Information Systems and Networks | Netcracker}
{\vspace{-0.3cm}}
%------------------------------------------------
\end{entrylist}


%----------------------------------------------------------------------------------------
%   WORK EXPERIENCE SECTION
%----------------------------------------------------------------------------------------

\section{Experience}

\begin{entrylist}
%------------------------------------------------

\entry
    {Jul 2021 -- Now}
    {Netcracker}
    {}
    {\jobtitle{Junior Software Developer}\\ 
    - Optimized and structured for better readability the code of an existing microservice written in Java by better organization of object dependencies;\\ - Fixed back-end and front-end defects in platfrom components;\\ - Maintained the working dev environments in Kubernetes by deploying vital microservices with properly configured parameters;\\ - Configured API for GraphQL queries in terms of new microservice.\\ - Developed unit tests for new Java component with high code coverage using JUnit5.\\

Intensively worked with:\\
- Java, Angular, SpringBoot, PostgreSQL\\
- Microservices, k8s, REST api, Postman\\
- Jenkins, Git, Maven}

\entrySkoltech
    {2021, Aug}
    {Skoltech Summer Internship}
    {
    \\
    Developed independent load balancer for iPerf-based 5G speedtest service by working with my colleague using Python Flask server in combination with Docker containerization and Swagger API.
    }

\entry
    {2020, Apr}
    {NTI Hackathon}
    {}
    {\jobtitle{Student stream, Wireless Technologies Profile}\\
Won first place out of 8 teams in hackathon at NTI by working with five colleagues to develop a noise-resistant algotirthm for optimal data transmission over a noisy channel.
    }

%\entry
%  {2018--Now}
%  {Talento ConCrédito}
%  {Culiacán, Sinaloa}
%  {\jobtitle{Developer}\\
%  Create, Modify, Maintain and deploy API's, in the develop, QA and production environment.
%Our API's help to provide a financial solutions simplify by technology.
%
%Work with the scrum methodology.
%
%%Version control of our projects using git.
%
%Use the Google Data Studio tool to create reports that represent indicators of the systems.
%
%Use Docker to help us through containers to implement tools in a matter of seconds without the need to %install anything. It's multiplatform, which guarantees that the availability of our application works independently of the operating system.}
%------------------------------------------------
%\entry
%  {2018--2018}
%  {ConCrédito }
%  {Cuiacán, Sinaloa}
%  {\jobtitle{Tester}
%  \begin{itemize}
%    \item Analyze systems functional models. 
%    \item Create test cases for the systems.
%    \item Run test cases.
%    \item Create autumatized tests using selenium.
%    \item Black-Box Testing
%    \item Record evidence and raise defects to the developers.
%  \end{itemize}}

\end{entrylist}


\section{Volunteering}

\begin{entrylist}

\entrySkoltech
    {2018--2021}
    {Experience As a Teacher}
    {\jobtitle{}\\
Taught electrodynamics and general physics for pupils at the summer ecological school.}

\end{entrylist}


%----------------------------------------------------------------------------------------
%   SOFT SKILLS SECTION
%----------------------------------------------------------------------------------------

%\section{Soft Skills}
%{\vspace{0.1cm}}
%I actually love working in team and interacting with other people. I'm not afraid to be creative and sometimes it leads to reinventing the wheel. Responsible and hardworking. I try to find the right approach to different people while working on problem.

%----------------------------------------------------------------------------------------
%       INTERESTS SECTION
%----------------------------------------------------------------------------------------

\section{Electrical Engineering Experience}
{\vspace{0.005cm}}
During hole my conscious life I tend to create different electrical devices and for the last 8 years I have been improving my skills in creating those. Starting with the very basics and Arduino, ending with STM32 and digital processing using FPGAs. During this time, I sought to understand the very essence of electronic engineering, designing and implementing printed circuit boards with my own hands. I managed to implement many interesting projects from scratch, some of which are now helping me to create complex projects brick by brick right now.

I have worked with many protocols such as SPI, I2C, UART, CAN, One-Wire, and also have extensive experience with timers, DMA, DAC, ADC. I have implemented a digital encoder algorithm for ultra-wideband noise-immune signal transmission, builded a flyback transformer with specified parameters, a mechanically scanned 3D display, a level detector based an accelerometer and many other small projects with interesting circuit solutions.
\vspace{0.1cm}

\section{Computer Architecture Experience}
{\vspace{0.005cm}}
I attended additional courses at my university that were aimed at a detailed analysis of the MIPS microarchitecture (Branch Prediction, Pipelining, Caches, etc.). Also, as part of this course, we completed practical tasks in the MIPS simulator. The course itself: https://mipt-ilab.github.io/mipt-mips/. 
\vspace{0.1cm}

\section{CISCO Computer Networks Experience}
{\vspace{0.005cm}}
I have a good theoretical basis for understanding computer networks (dynamic routing protocols, OSI model levels, network architecture) with practical reinforcement by working in GNS3 and on real CISCO equipment.
\vspace{0.1cm}

\section{Interests}
%------------------------------------------------
%\entry
%{Проф-ные}
%{- {\normalfont Continue learning, work and develop a career in what I'm passionate about, which are %all the information technologies. }}
%{}
%{\vspace{-0.3cm}}
%------------------------------------------------
{\vspace{0.05cm}}
I am in love with modern digital technologies in term of creating something new. Programming is not the only way for me to influence this entire world, electrical engineering also allows me to create new unique things. This is my hobby and I love to share my knowledge with those who are also passionate about the same thing as me. I actually love working in team and interacting with other people -- it makes me feel together we build something incredible and this is amazing. 
%------------------------------------------------

\end{document}